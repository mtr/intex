% \iffalse meta-comment
% $Id: intex.dtx 34 2005-10-25 11:31:35Z mtr $
% Copyright (C) 2005 by Martin Thorsen Ranang <intex@ranang.org>
% -------------------------------------------------------
% 
% This file may be distributed and/or modified under the
% conditions of the LaTeX Project Public License, either version 1.2
% of this license or (at your option) any later version.
% The latest version of this license is in:
%
%    http://www.latex-project.org/lppl.txt
%
% and version 1.2 or later is part of all distributions of LaTeX 
% version 1999/12/01 or later.
%
% \fi
%
% \iffalse
%<*driver>
\ProvidesFile{intex.dtx}
%</driver>
%<package>\NeedsTeXFormat{LaTeX2e}[1999/12/01]
%<package>\ProvidesPackage{intex}
%<*package>
   [2005/08/05 v1.0 A concept indexing and typesetting package]
%</package>
%<*batchfile>
\begingroup
%%
%% Copyright (C) 2005 by Martin Thorsen Ranang <intex@ranang.org>
%%
%% This file may be distributed and/or modified under the conditions of
%% the LaTeX Project Public License, either version 1.2 of this license
%% or (at your option) any later version.  The latest version of this
%% license is in:
%% 
%%    http://www.latex-project.org/lppl.txt
%% 
%% and version 1.2 or later is part of all distributions of LaTeX version
%% 1999/12/01 or later.
%%

\input docstrip.tex
\keepsilent

\usedir{tex/latex/intex}

\preamble

This is a generated file.

Copyright (C) 2005 by Martin Thorsen Ranang <intex@ranang.org>

This file may be distributed and/or modified under the conditions of
the LaTeX Project Public License, either version 1.2 of this license
or (at your option) any later version.  The latest version of this
license is in:

   http://www.latex-project.org/lppl.txt

and version 1.2 or later is part of all distributions of LaTeX version
1999/12/01 or later.

\endpreamble

\askforoverwritefalse
\generate{\file{intex.sty}{\from{intex.dtx}{intex}}}

\obeyspaces
\Msg{*************************************************************}
\Msg{*                                                           *}
\Msg{* To finish the installation you have to move the following *}
\Msg{* file into a directory searched by TeX:                    *}
\Msg{*                                                           *}
\Msg{*     intex.sty                                             *}
\Msg{*                                                           *}
\Msg{* To produce the documentation run the file intex.dtx       *}
\Msg{* through LaTeX.                                            *}
\Msg{*                                                           *}
\Msg{* Happy TeXing!                                             *}
\Msg{*                                                           *}
\Msg{*************************************************************}

\endgroup
%</batchfile>
%
%<*driver>
\documentclass{ltxdoc}
\usepackage{intex}[2005/08/05]
\usepackage[T1]{fontenc}
\usepackage[latin1]{inputenc}
\usepackage[norsk,english]{babel}
\usepackage{fancyvrb}
\usepackage{url}
\usepackage{svn}
\SVN $Rev: 34 $
\SVN $Date: 2005-10-25 13:31:35 +0200 (tir, 25 okt 2005) $
\EnableCrossrefs
%\CodelineIndex
\RecordChanges
\begin{document}
  \DocInput{intex.dtx}
  %\PrintChanges
  %\PrintIndex
\end{document}
%</driver>
% \fi
%
% \CheckSum{0}
%
% \CharacterTable
%  {Upper-case    \A\B\C\D\E\F\G\H\I\J\K\L\M\N\O\P\Q\R\S\T\U\V\W\X\Y\Z
%   Lower-case    \a\b\c\d\e\f\g\h\i\j\k\l\m\n\o\p\q\r\s\t\u\v\w\x\y\z
%   Digits        \0\1\2\3\4\5\6\7\8\9
%   Exclamation   \!     Double quote  \"     Hash (number) \#
%   Dollar        \$     Percent       \%     Ampersand     \&
%   Acute accent  \'     Left paren    \(     Right paren   \)
%   Asterisk      \*     Plus          \+     Comma         \,
%   Minus         \-     Point         \.     Solidus       \/
%   Colon         \:     Semicolon     \;     Less than     \<
%   Equals        \=     Greater than  \>     Question mark \?
%   Commercial at \@     Left bracket  \[     Backslash     \\
%   Right bracket \]     Circumflex    \^     Underscore    \_
%   Grave accent  \`     Left brace    \{     Vertical bar  \|
%   Right brace   \}     Tilde         \~}
%
%
% \changes{v1.0}{2005/08/05}{Initial version}
%
% \GetFileInfo{intex.dtx}
%
% \DoNotIndex{\newcommand,\newenvironment}
%
% ^^A From ltugboat.cls
%
% ^^A Typeset the name of an environment
% \providecommand\env[1]{\textsf{#1}}
% \providecommand\clsname[1]{\textsf{#1}}
% \providecommand\pkgname[1]{\textsf{#1}}
% \providecommand\optname[1]{\textsf{#1}}
% \providecommand\progname[1]{\textsf{#1}}
%
% ^^A A list of options for a package/class
% \newenvironment{optlist}{\begin{description}%
%   \renewcommand\makelabel[1]{%
%     \descriptionlabel{\mdseries\optname{##1}}}%
%   \itemsep0.25\itemsep}%
%  {\end{description}}
%
% ^^A Utility macros
%
% ^^A Special dashes
% \def\thinskip{\hskip 0.16667em\relax}
% \def\endash{--}
% \def\emdash{\endash-}
% \def\d@sh#1#2{\unskip#1\thinskip#2\thinskip\ignorespaces}
% \def\dash{\d@sh\nobreak\endash}
% \def\Dash{\d@sh\nobreak\emdash}
%
% ^^A Example macros - adapted from the `fvrb-ex' package
% ^^A ---------------------------------------------------
%
% ^^A Take care that we use here the "Z" character as comment character,
% ^^A to avoid to use an 8 bit one which can cause portability problems.
% ^^A But we can't use any more the "Z" into the example environments
% ^^A of this documentation.
%
% \makeatletter
% \newcommand{\BeginExample}[1][0]{%
% \parindent=0pt
% \multiply\topsep by 2
% \VerbatimEnvironment
% \begin{VerbatimOut}[gobble=#1]{\jobname.tmp}}
%
% \newcommand{\BelowExample}[1]{%
% \VerbatimInput[gobble=4,commentchar=Z,numbersep=3pt,frame=single,
%                numbers=left]{\jobname.tmp}
% \catcode`\Z=9\relax%
% #1\par}
%
% \newcommand{\SideBySide@Example}[1]{%
% \@tempdimb=\FV@XRightMargin
% \advance\@tempdimb -5mm
% \vspace{2mm}
% \begin{minipage}[c]{\@tempdimb}
%   \fvset{xrightmargin=0pt}
%   \catcode`\Z=9\relax%
%   #1
% \end{minipage}%
% \@tempdimb=\textwidth
% \advance\@tempdimb -\FV@XRightMargin
% \advance\@tempdimb 5mm
% \begin{minipage}[c]{\@tempdimb}
%   \VerbatimInput[commentchar=Z,numbersep=3pt,frame=single,
%                  numbers=left,xleftmargin=5mm,xrightmargin=0pt]{\jobname.tmp}
% \end{minipage}
% \vspace{2mm}}
%
% \def\Example{%
% \catcode`\^^M=\active
% \@ifnextchar[{\catcode`\^^M=5\Example@}{\catcode`\^^M=5\Example@@}}
% \def\Example@[#1]{\fvset{#1}\Example@@}
% \def\Example@@{\BeginExample}
% \def\endExample{%
% \end{VerbatimOut}%
% \BelowExample{\input{\jobname.tmp}}}
%
% \def\CenterExample{%
% \catcode`\^^M=\active
% \@ifnextchar[{\catcode`\^^M=5\CenterExample@}
%              {\catcode`\^^M=5\CenterExample@@}}
% \def\CenterExample@[#1]{\fvset{#1}\CenterExample@@}
% \def\CenterExample@@{\BeginExample}
% \def\endCenterExample{%
% \end{VerbatimOut}%
% \center
% \BelowExample{\input{\jobname.tmp}}
% \endcenter}
% 
% \def\SideBySideExample{%
% \catcode`\^^M=\active
% \@ifnextchar[{\catcode`\^^M=5\SideBySideExample@}%
%              {\catcode`\^^M=5\SideBySideExample@@}}
% \def\SideBySideExample@[#1]{\fvset{#1}\SideBySideExample@@}
% \def\SideBySideExample@@{\BeginExample[4]}
% \def\endSideBySideExample{%
% \end{VerbatimOut}%
% \SideBySide@Example{\input{\jobname.tmp}}}
% \makeatother
%
% ^^A End of example macros from `fvrb-ex'
%
% \makeatletter
% \date{\the\year/\two@digits{\the\month}/\two@digits{\the\day}}
% \makeatother
%
% ^^A The beginning of the documentation itself.
%
% \title{The \textsf{\InTeX} package\thanks{This document
%   corresponds to \textsf{intex} revision~\SVNRev, dated \today.}}
% \author{Martin Thorsen Ranang \\ \texttt{intex@ranang.org}}
%
% \maketitle
%
% \section{Introduction}

% This package adds functionality to \LaTeX\ that eases typesetting
% and indexing of phrases, acronyms and names in a consistent manner.

% \DescribeMacro{intex} \DescribeMacro{\co}
% The really short usage description is that in order to use the
% package, insert |\usepackage{intex}| at the beginning of your
% \LaTeX\ source file.  After that, you can wrap the macro
% |\co{|\meta{concept}|}| around any concept you want to typeset
% and/or typeset in some special way.

% \section{Background}

% I've been using \LaTeX\ since the spring of 1997.  Since then I've
% written several technical documents.  I try to present my work in an
% accessible way.  Additionaly, I believe that \LaTeX\ can help with
% the technicalities of presenting technical writing in a clear and
% precise way.  For example, I've tried to always explain every
% acronym used (the first time it occurrs in a document), include a
% meaningful index, and also to leave some typeset clues to the
% reader, so that it will be easier to find the key phrases in the
% document.

% Already there exists packages that provide functionality that eases
% the \DescribeMacro{acronym}acronym\footnote{The |acronym| package,
% written by Tobias Oetiker, available from
% \url{CTAN:/macros/latex/contrib/acronym}.} and
% \DescribeMacro{index}indexing\footnote{The |index| package, written
% by David M.\,Jones, available from
% \url{CTAN:/macros/latex/contrib/index}.} operations mentioned above.
% However, problems quickly arise when writing about an acronym in
% both singular and plural.  For example, let's say you want to use
% the concept \emph{informed search} (abbreviated IS).  Then, if you
% want to write about that concept in plural, the logical acronym
% would be ISes (informed searches).  At the same time, you probably
% want those two occurences---perhaps typeset several chapters
% apart---to be indexed as being the same concept.

% The \InTeX\ package was written to reduce the work needed to handle
% such a task.  This has been done by combining the functionality of
% the |acronym| and the |index| packages with an external
% \co{Python}\footnote{\co{Python} is available from
% \url{http://www.python.org/}.} script.

%
% \section{Usage}

% How you can use \InTeX\ should be clearer after examining some
% examples.  The central idea in \InTeX\ is that a phrase or a word
% worth indexing constitutes some kind of a \co{concept}---in a broad
% sense of the word.  A \co{concept} can be either an \co{acronym} (or
% abbreviation), the name of an entity (like a \co{person} or an
% organization), or of the ``plain'' kind (simply a phrase).  Hence,
% we will refer to the three kinds of \co{concepts} as
% \emph{\co{acronym}}, \emph{\co{person}}, and \emph{\co{plain}}.

% In the above paragraph, the word ``concepts'' was defined as a
% \co{concept} of the \co{plain} kind, and it was defined to be
% indexed as the word ``concept''.  Furthermore, the words
% ``acronym'', ``person'', and ``plain'' where also defined as
% \co{plain concepts}.  However, these concepts are defined as
% \co{sub-concepts} of ``concept'' and should be indexed accordingly.

% \subsection{Package Options}
% \begin{optlist}
%   \item[noindex]: Whether \InTeX\ should generate an index or not.
% (\emph{Default: true}).
% \item[nowarnundef]: Whether \InTeX\ should generate in-document
%  warnings where unknown/undeclared concepts are encountered.
% (\emph{Default: true}).
% \item[nomargin\meta{type}]: Tell \InTeX\ not to add margin notes
%  whenever \emph{new} concepts of kind \meta{type} are typeset, where
%  \meta{type} is one of \optname{plain}, \optname{acronym}, or
%  \optname{person}.

% \end{optlist}

% \subsection{Examples}

% \begin{SideBySideExample}[xrightmargin=5cm]
%   \makeatletter
%   \@itx@margin@acronymfalse%
%   \makeatother
%   It is easy to refer to (and thus index)
%   acronyms, like \co{H2O}.  And 
%   sub-concepts, like \co{H2O-reserve}.
% \end{SideBySideExample}
%

% \begin{SideBySideExample}[xrightmargin=5cm]
%   \makeatletter
%   \@itx@margin@acronymfalse%
%   \makeatother
%   We could talk about multiple \co{IDs}, 
%   or a single \co{ID}. 
% \end{SideBySideExample}

% In this section, let's assume that we would like to write
% about the \co{\InTeX\ logo}.  This might be a good time to go drink
% some \co{H2O}.

% Example text: I believe that the problem of generating good
% indices is \co{AIC}.  

% \subsubsection{Index Definitions}
% The concept index definition file used for the above paragraph looks
% like:
% \VerbatimInput[showtabs=true,tabsize=4,numbers=left,fontsize=\small,lastline=1]{intex.itx}
% The file can be divided into different sections, according to the
% kind of concepts to be declared.  To set the
% current section, use a single line that must contain exactly 
% |"% *|\meta{type}|*"|, where \meta{type} is either |ACRONYMS|,
% |CONCEPTS|, or |PERSONS|.

% \VerbatimInput[showtabs=true,tabsize=4,numbers=left,fontsize=\small,firstline=2,lastline=10]{intex.itx}
% the above line should mean that |\co{synsets}| in the text should be
% indexed as if it read |\co{synset}|.  However, if the plural of the
% concept occurs first (in the [part of] document), its full-form
% should be \emph{typeset} ``synonym set'' + ``s'' (indicated by the |#y|) in
% the index file.  In other words, it only a short-hand notation.  On the
% other hand, in the next definition, another short-hand notation |#y|
% is used that will transform ``y'' into ``ies'' as the end of the last
% word:
%
% \VerbatimInput[showtabs=true,tabsize=4,numbers=left,fontsize=\small,firstline=11]{intex.itx}

% \subsection{Compilation}
% \label{sec:compilation}
% \DescribeMacro{makeintex} As mentioned earlier, the package includes
% an external program named |makeintex|.  The typical usage of
% |makeintex|, given that your document is named \meta{name}, would
% be:
% 
% \begin{enumerate}
% \item |latex |\meta{name}|.tex|
% \item |makeintex |\meta{name} \meta{name}|.itx -o |\meta{name}|.rix [-a acronyms.tex -p persons.tex]|
% \item |makeindex |\meta{name}
% \item |makeindex -o |\meta{name}|.rid |\meta{name}|.rix|
% \item |latex |\meta{name}|.tex|
% \end{enumerate}

% \section{Macros}
%
% \DescribeMacro{\InTeX}
% This is simply a macro for typesetting the \co{\InTeX\ logo}. 
%
% \StopEventually{}
%
% \section{Implementation}
%
% After the customary identification,
%    \begin{macrocode}
\def\filename{intex}%
\ProvidesPackage{intex}[2005/08/05 v0.1
Support for concept, acronym, and proper-name typesetting and indexing]%
%    \end{macrocode}
% we continue by defining the package options.
%
% \subsection{Package Options}
%
% \begin{macro}{noindex}
% \begin{macro}{\if@itx@index}
% Let the conditional |\if@itx@index| control whether
% \InTeX\ should generate an index or not.  The default is to perform
% indexing.  The option |noindex| turns this feature off.
%    \begin{macrocode}
\newif\if@itx@index%
\@itx@indextrue%
\DeclareOption{noindex}{\@itx@indexfalse}%
%    \end{macrocode}
% \end{macro}
% \end{macro}
%
% \begin{macro}{nowarnundef}
% \begin{macro}{\if@itx@nowarnundef}
% The conditional |\if@itx@nowarnundef| controls whether
% \InTeX\ should include in-document warnings about undefined concepts
% or not.  The default is to warn about undefined concepts inside the
% document.  The |nowarnundef| option turns this feature off.
%    \begin{macrocode}
\newif\if@itx@warn@undef%
\@itx@warn@undeftrue%
\DeclareOption{nowarnundef}{\@itx@warn@undeffalse}%
%    \end{macrocode}
% \end{macro}
% \end{macro}

% \begin{macro}{nomarginplain}
% \begin{macro}{nomarginacronym}
% \begin{macro}{nomarginperson}
% \begin{macro}{\if@itx@margin@plain}
% \begin{macro}{\if@itx@margin@acronym}
% \begin{macro}{\if@itx@margin@person}

% The conditionals |\if@itx@margin@|\meta{kind}---where \meta{kind} is
% one of |plain|, |acronym|, and |person|---control whether the short
% version of each first-occurrence of a concept (of kind \meta{kind},
% per significant document part) should also be typeset as a
% margin-label.
%
%    \begin{macrocode}
\newif\if@itx@margin@plain%
\newif\if@itx@margin@acronym%
\newif\if@itx@margin@person%
\@itx@margin@plaintrue%
\@itx@margin@acronymtrue%
\@itx@margin@persontrue%
\DeclareOption{nomarginplain}{\@itx@margin@plainfalse}
\DeclareOption{nomarginacronym}{\@itx@margin@acronymfalse}
\DeclareOption{nomarginperson}{\@itx@margin@personfalse}
%    \end{macrocode}
% \end{macro}
% \end{macro}
% \end{macro}
% \end{macro}
% \end{macro}
% \end{macro}
%
% Next, process the options.
%    \begin{macrocode}
\ProcessOptions
%    \end{macrocode}
%
% \subsection{External Packages}
%
% \begin{macro}{index}
% Now, if |\if@itx@index| is \emph{true}, then require the package
% |index| to be loaded.  If not, we define a handy macro usually
% defined in that package.
%    \begin{macrocode}
\if@itx@index%
  \RequirePackage{index}%
  \makeindex%
  \newindex{raw}{rix}{rid}{Raw Index}%
\else%
  \def\@nearverbatim{\expandafter\strip@prefix\meaning}%
\fi
%    \end{macrocode}
% \end{macro}

% \begin{macro}{acronym}
% Anyhow, require the |acronym| package to be loaded.
%    \begin{macrocode}
\RequirePackage{acronym}
%    \end{macrocode}
% \end{macro}

% \subsection{The \InTeX\ Logo}
% \begin{macro}{\InTeX}
% Define a \TeX-ish logo for this package.
%    \begin{macrocode}
\newcommand*{\InTeX}{\textsl{In}\kern-.07em\TeX}%
%    \end{macrocode}
% \end{macro}


% \subsection{Font Definitions}

% The following commands define the font-selection commands used to
% typeset the different kinds of concepts in different situations.

% \begin{macro}{\itxplaindeffont}
% \begin{macro}{\itxplainfollowfont}
% \begin{macro}{\itxplainmarginfont}
% These commands are used to typeset plain concepts.
%    \begin{macrocode}
\newcommand{\itxplaindeffont}[1]{\emph{#1}}
\newcommand{\itxplainfollowfont}[1]{#1}
\newcommand{\itxplainmarginfont}[1]{\raggedright\footnotesize\textsf{#1}}
%    \end{macrocode}
% \end{macro}
% \end{macro}
% \end{macro}

% \begin{macro}{\itxacronymdeffont}
% \begin{macro}{\itxacronymdefshortfont}
% \begin{macro}{\itxacronymshortfont}
% \begin{macro}{\itxacronymmarginfont}

% For acronyms:
%    \begin{macrocode}
\newcommand{\itxacronymdeffont}[1]{#1}
\newcommand{\itxacronymdefshortfont}[1]{\emph{#1}}
\newcommand{\itxacronymshortfont}[1]{#1}
\newcommand{\itxacronymmarginfont}[1]{\raggedright\footnotesize\textsf{#1}}
%    \end{macrocode}
% \end{macro}
% \end{macro}
% \end{macro}
% \end{macro}

% \begin{macro}{\itxpersondeffont}
% \begin{macro}{\itxpersonfirstfont}
% \begin{macro}{\itxpersonlastfont}
% \begin{macro}{\itxpersonmarginfont}

% For persons:
%    \begin{macrocode}
\newcommand{\itxpersondeffont}[1]{\emph{#1}}
\newcommand{\itxpersonfirstfont}[1]{#1}
\newcommand{\itxpersonlastfont}[1]{#1}
\newcommand{\itxpersonmarginfont}[1]{\raggedright\footnotesize\textsf{#1}}
%    \end{macrocode}
% \end{macro}
% \end{macro}
% \end{macro}
% \end{macro}

% \subsection{The (Low-Level) Clockwork of the Package}
%
% \begin{macro}{\co@serial}
% First, define a counter that is used to enumerate new concept
% definitions.
%    \begin{macrocode}
\newcounter{co@serial}%
\newcounter{co@equiv@serial}%
\newcounter{co@type}%
%    \end{macrocode}
% \end{macro}
%
% \begin{macro}{\itxundefcomment}
% Then, define the comment to display where use of undefined concepts
% are detected.
%    \begin{macrocode}
\newcommand*\itxundefcomment[1]{\emph{(undefined concept ``#1'')}}
%    \end{macrocode}
% \end{macro}

% Define a couple of convenience macros.
%    \begin{macrocode}
\long\def\@firstofthree#1#2#3{#1}%
\long\def\@secondofthree#1#2#3{%
  %\PackageWarning{@secondofthree}{#1}%
  #2}%
%\newcommand*\@secondofthree[3]{#2}%
%    \end{macrocode}

% Make it possible to reset the ``defined'' flag for each concept.
% After a reset, the next time that concept occurs, it is typeset as
% if it's the first occurrence of that concept.
%    \begin{macrocode}
\def\ITX@reset#1{%
  \global\expandafter\let\csname itx@#1\endcsname\relax}
%    \end{macrocode}

% \subsubsection{Typesetting of Margin Labels}
% \begin{macro}{\@itxmarginlabel}

% Define a macro to typeset the concepts at first-occurrence points in
% the margin.

%    \begin{macrocode}
\newcommand*\@itxmarginlabel[2]{%
  \hspace{0pt}%
%    \end{macrocode}
  %
  % The second argument is the \emph{identity} of the entity we're
  % typesetting, while the first argument signals whether we're
  % typesetting a\ldots
  %
%    \begin{macrocode}
  \PackageWarning{marginlabel}{Got (\meaning#1, \meaning#2)}%
  \ifcase#1%
%    \end{macrocode}
  % \ldots plain concept, \ldots
%    \begin{macrocode}
    \if@itx@margin@plain%
      \marginpar{\itxplainmarginfont{\ITX@itxs{#1}{#2}}}%
    \fi%
  \or%
%    \end{macrocode}
  % \ldots an acronym, \ldots
%    \begin{macrocode}
    \if@itx@margin@acronym%
      \marginpar{\itxacronymmarginfont{\ITX@itxs{#1}{#2}}}%
    \fi%
  \or%
%    \end{macrocode}
  % \ldots or a person's name.
%    \begin{macrocode}
    \if@itx@margin@person%
      \marginpar{\itxpersonmarginfont{\ITX@itxl{#1}{#2}}}%
    \fi%
  \fi%
}
%    \end{macrocode}
% \end{macro}
% \begin{macro}{\ITX@used}
% Value to flag a concept as used.
%    \begin{macrocode}
\newcommand*\ITX@used{@<>@<>@}
%    \end{macrocode}
% \end{macro}

% \begin{macro}{\ITX@get}
%    \begin{macrocode}
\newcommand*\ITX@get[2]{%
  %\PackageWarning{ITX@get}{#1}%
  \ifx#1\relax%
  \else%
    \expandafter#2#1%
  \fi%
}
%    \end{macrocode}
% \end{macro}

% \begin{macro}{\itxplainarea}
% \begin{macro}{\itxacronymarea}
% \begin{macro}{\itxpersonarea}
% \begin{macro}{\@itxarea}
% Significant-area definitions.  When these counters change, the
% concepts concerned will be typeset as first occurrences.
%    \begin{macrocode}
\newcommand*\itxplainarea{\thesubparagraph.\thepage}%
\newcommand*\itxacronymarea{\thesubparagraph.\thepage}%
\newcommand*\itxpersonarea{\thesubsubsection}%
\newcommand*\@itxarea[1]{%
  \ifcase#1%
    \itxplainarea%
  \or%
    \itxacronymarea%
  \or%
    \itxpersonarea%
  \fi%
}
%    \end{macrocode}
% The default (empty) area definitions.
%    \begin{macrocode}
\def\itx@last@pos0{}%
\def\itx@last@pos1{}%
\def\itx@last@pos2{}%
%    \end{macrocode}
% \end{macro}
% \end{macro}
% \end{macro}
% \end{macro}

%    \begin{macrocode}
\newcommand*\ITX@itxs[2]{%
  %\PackageWarning{ITX@itxs}{Got (\meaning#1, \meaning#2)}%
  %\edef\value{\csname fn\number#1@#2\endcsname}%
  %\PackageWarning{ITX@itxs:2}{value = \meaning\value = \value}%
  %\csname fnss@\number\value\endcsname%
  \csname fnss@\number#2\endcsname%
}
\newcommand*\ITX@itxl[2]{%
  %\edef\value{\csname fn#1@#2\endcsname}%
  %\csname fnsl@\number\value\endcsname%
  \csname fnsl@\number#2\endcsname%
}
\newcommand*{\itxs}[2]{%
  \texorpdfstring{\protect\@itxs{#1}{#2}}{#1}}
\newcommand*{\@itxs}[2]{%
  \ifcase\number#1%
%    \end{macrocode}
    % Plain.
%    \begin{macrocode}
    \itxplainfollowfont{\ITX@itxs{#1}{#2}}%
  \or%
%    \end{macrocode}
    % Acronym.
%    \begin{macrocode}
    \itxacronymshortfont{\ITX@itxs{#1}{#2}}%
  \or%
%    \end{macrocode}
    % Person.
%    \begin{macrocode}
    \itxpersonlastfont{\ITX@itxl{#1}{#2}}%
  \fi%
}
\newcommand*{\itxl}{\protect\@itxl}
\newcommand*{\@itxl}[2]{%
   \ITX@itxl{#1}{#2}%
 }
\newcommand*{\itxf}[2]{%
  \texorpdfstring{\protect\@itxf{#1}{#2}}{\ITX@itxl{#1}{#2} (#1)}%
 }
\newcommand*{\@itxf}[2]{%
  \ifcase\number#1%
%    \end{macrocode}
    % Plain.
    % Typeset margin-notes if applicable.
%    \begin{macrocode}
    \PackageWarning{ZZ}{plain}%
    \PackageWarning{ZZ}{@itxmarginlabel:enter}%
    \@itxmarginlabel{#1}{#2}%
    \PackageWarning{ZZ}{@itxmarginlabel:exit}%
%    \end{macrocode}
    % Typeset the concept.
%    \begin{macrocode}
    \itxplaindeffont{\ITX@itxs{#1}{#2}}\nolinebreak %
  \or%
%    \end{macrocode}
    % Acronym.
    % Typeset the concept.  Note in-between margin label.
%    \begin{macrocode}
    \PackageWarning{ZZ}{acronym}%
    \itxacronymdeffont{%
      \PackageWarning{ZZ}{ITX@itxl:enter}%
      \ITX@itxl{#1}{#2} %
      \PackageWarning{ZZ}{ITX@itxl:exit}%
      %\nolinebreak[3] %
%    \end{macrocode}
      % Typeset margin-notes if applicable.
%    \begin{macrocode}
      \@itxmarginlabel{#1}{#2}%
%    \end{macrocode}
      % Continue typesetting the concept.
%    \begin{macrocode}
      \itxacronymdefshortfont{%
        \itxacronymshortfont{(\ITX@itxs{#1}{#2})}}%
    }%
  \or%
%    \end{macrocode}
    % Person.
    % Typeset the concept.  Note in-between margin label.
%    \begin{macrocode}
    \PackageWarning{ZZ}{person}%
    \itxpersondeffont{%
      \itxpersonfirstfont{%
        \ITX@itxs{#1}{#2}} %
      %\nolinebreak[3] %
%    \end{macrocode}
      % Typeset margin-notes if applicable.
%    \begin{macrocode}
      \@itxmarginlabel{#1}{#2}%
%    \end{macrocode}
      % Continue typesetting the concept.
%    \begin{macrocode}
      \itxpersonlastfont{%
        \ITX@itxl{#1}{#2}%
      }%
    }%
  \fi%
%    \end{macrocode}
  % Now, do the used/unused accounting.
%    \begin{macrocode}
  \expandafter\ifx\csname itx@#2\endcsname\ITX@used%
    %\relax%
  \else%
    \global\expandafter\let\csname itx@#2\endcsname\ITX@used%
    %\ITX@addtoclearlist{#2}% MTR
  \fi%
  %\ITX@logged{#2} MTR
}%
%    \end{macrocode}
%
% \begin{macro}{\@itxrecordarea}
% A macro used to update the current used/non-used status of each
% concept.  This macro only use the \emph{type} and \emph{numeric
% id} of the concept.
%    \begin{macrocode}
\newcommand*{\@itxrecordarea}[2]{%
%    \end{macrocode}
  % Record this area:
%    \begin{macrocode}
  \edef\curr@pos{\@itxarea{#1}}%
%    \end{macrocode}
  % Remember the last area where this concept (second argument) was used.
%    \begin{macrocode}
  \PackageWarning{plain:x1}{Got (\meaning#1, \meaning#2)}%
  \edef\last@pos{\csname itx@last@pos#1@#2\endcsname}%
  \ifx\curr@pos\last@pos%
%    \end{macrocode}
    % We're still in the same area.  Hence, we do nothing.
%    \begin{macrocode}
  \else%
%    \end{macrocode}
    % The area has changed.
%    \begin{macrocode}
    \ITX@reset{#2}%
    \PackageWarning{plain:x2}{reset}%
  \fi%
  \expandafter\xdef\csname itx@last@pos#1@#2\endcsname{\curr@pos}%
  \PackageWarning{plain:x3}{exported position}%
}
%    \end{macrocode}
%  \end{macro}
%
% \begin{macro}{\@itxplain}
%    \begin{macrocode}
\newcommand*{\@itxplain}[2]{%
%    \end{macrocode}
% First, update the ``last used'' status of the current concept so
% that it refers to the current area.
%    \begin{macrocode}
  \@itxrecordarea{#1}{#2}%
%    \end{macrocode}
  % Then, typeset the concept.
%    \begin{macrocode}
  \PackageWarning{ZZ}{itxplain:enter}%
  \expandafter\ifx\csname itx@#2\endcsname\ITX@used%
    \PackageWarning{ZZ}{itxs:enter}%
    \itxs{#1}{#2}%
    \PackageWarning{ZZ}{itxs:exit}%
  \else%
    \PackageWarning{ZZ}{itxf:enter}%
    \itxf{#1}{#2}%
    \PackageWarning{ZZ}{itxf:exit}%
  \fi%
  \PackageWarning{ZZ}{itxplain:exit}%
}
%    \end{macrocode}
% \end{macro}
%
% \begin{macro}{\@itxalias}
% Define the id of the equivalent entry.
% Get the id of the main index entry for which this is an alias.
% Keep the original definition as |\@orig|.
% Redefine the main entry (as in
% |\expandafter\gdef\csname fn@#1\endcsname{{#2}{#2}}|.)
% Now, typeset the alias by using the main index entry ID.
% Finally, reset the definition of the main entry.
%    \begin{macrocode}
\newcommand*{\@itxalias}[2]{%
  \PackageWarning{x0}{args = (#1, "#2")}%
  %\edef\equiv{\csname fn\number#1e@#2\endcsname}%
  %\PackageWarning{x1}{value(equiv): "\equiv", meaning: "\meaning\equiv"}%
  \edef\@mainserial{\expandafter\@firstoftwo#2}%
  \edef\@equivserial{\expandafter\@secondoftwo#2}%
  \PackageWarning{x2}{main: "\@mainserial", equiv: "\@equivserial"}%
  % Record usage of the main concept entry.
  %\@itxrecordarea{#1}{\@mainserial}%
  %
  \@itxplain{#1}{\@equivserial}%
}%
%    \end{macrocode}
% \end{macro}
%
% \begin{macro}{\@itx@fakeindex}
%    \begin{macrocode}
\newcommand{\@itx@fakeindex}[1]{%
  \begingroup%
    \edef\@tempa{%
      \write\@auxout{%
        \string\@writefile{raw}{%
          \string\indexentry{#1}%
          {0}%
        }%
      }%
    }%
  \expandafter\endgroup\@tempa%
}
%    \end{macrocode}
% \end{macro}
% \begin{macro}{\co}

% The |\co| command is the only command the user should need to use.
% The usage is |\co{|\meta{identity}|}|, where \meta{identity} refers
% to a concept.  The idea is that the |\co{}| command should be
% wrapped around every concept the user want to either typeset or
% index in a special and consistent way or both.  Note that |\co| is
% a wrapper for the |\@itx| command.

%    \begin{macrocode}
\newcommand{\co}{\protect\@itx}
%    \end{macrocode}
% \end{macro}
% \begin{macro}{\@itx}
% The |\@itx| command works just as described for |\co| above.
%    \begin{macrocode}
\newcommand*{\@itx}[1]{%
  \def\@tempa{#1}%
%    \end{macrocode}
  % Handles, e.g., backslashes.
%    \begin{macrocode}
  \edef\@tempb{\@nearverbatim\@tempa}%
  %\PackageWarning{itx}{tempa = \meaning\@tempa, tempb = \meaning\@tempb}%
%    \end{macrocode}
%
% If \InTeX\ should generate an index, simply use the |index| package
% to write the identifying index entry.  The intricacies of how this
% concept should be indexed is handled externally by |makeintex|.  If
% no index should be generated, we still need some output for
% |makeintex| to work with, but we don't care where in the document
% each entry was typeset.  Hence, in that case an index entry with
% \emph{page} equal to 0 will be written instead.
%
%    \begin{macrocode}
  %\PackageWarning{ITXindex}{index:enter}%
  \if@itx@index%
    \index[raw]{#1}%
  \else%
    \@itx@fakeindex{\@tempb}%
  \fi%
  %\PackageWarning{ITXindex}{index:exit}%
%    \end{macrocode}
%
% Define a new conditional, |\iffound|, to signal whether the
% \meta{identity} is found.  Also define a new counter $n$.  Then, let
% $n$ loop through the values $\{0, 1, 2\}$.
%
%    \begin{macrocode}
  \setcounter{co@type}{0}%
  \newif\iffound%
  \loop\ifnum\theco@type<3%
%    \end{macrocode}
  % Check to see if the \meta{identity} is an (acronym or person)
  % \emph{alias} or a \emph{main} entry.  If it is an \emph{alias},
  % there exists a variable named |fn|$n$|e@|\meta{identity}.
  % However, if \meta{identity} refers to a \emph{main} entry, a
  % variable named |fn|$n$|@|\meta{identity} exists.  That is, without
  % the |e| before the |@|.  Also, it should be noted that no
  % \emph{alias} can exists without a main entry with the same
  % \meta{identity}.
  %
  % If an expansion of \meta{identity} is found, typeset it
  % accordingly and flag the finding by setting |\iffound|.
%    \begin{macrocode}
    \expandafter\ifx\csname fn\number\theco@type e@\@tempb\endcsname\relax%
      \expandafter\ifx\csname fn\number\theco@type @\@tempb\endcsname\relax\else%
        \PackageWarning{here}{plain found \meaning\@tempb}%
        \foundtrue%
        %\@itxplain{\number\theco@type}{\@tempb}%
	\edef\co@id{\csname fn\number\theco@type @\@tempb\endcsname}%
	\@itxplain{\number\theco@type}{\co@id}%
      \fi%
    \else%
      \PackageWarning{here}{alias found \meaning\@tempb}%
      %\@itxalias{\number\theco@type}{\@tempb}%
      \edef\co@id{\csname fn\number\theco@type e@\@tempb\endcsname}%
      \@itxalias{\number\theco@type}{\co@id}%
      \foundtrue%
    \fi%
%    \end{macrocode}
    % Increase $n$ by 1 and perform a new iteration of the loop.
%    \begin{macrocode}
    \stepcounter{co@type}%
  \repeat%
%    \end{macrocode}
    % If no expansion of \meta{identity} could be found, warn the
    % user.  Furthermore, an in-document warning will be typeset if
    % |\if@itx@warn@undef| is \emph{true}.
%    \begin{macrocode}
  \iffound%
  \else%
    \PackageWarning{InTeX}{Concept `#1' is not defined}%
    \if@itx@warn@undef%
      \textbf{\itxundefcomment{#1}}%
    \else%
      #1%
    \fi%
  \fi%
}
%    \end{macrocode}
% \end{macro}
%
% \begin{macro}{\personused}
%    \begin{macrocode}
\newcommand*{\personused}[1]{%
  \expandafter\ifx\csname pnused@#1\endcsname\PN@used%
    \relax%
  \else%
    \global\expandafter\let\csname pnused@#1\endcsname\PN@used%
    \global\let\PN@populated\PN@used%
  \fi%
}
%    \end{macrocode}
% \end{macro}
%
% \begin{macro}{\@itxdefineforms}
%    \begin{macrocode}
\newcommand\@itxdefineforms[3]{%
  \expandafter\gdef\csname fnss@\number#1\endcsname{#2}%
  \expandafter\gdef\csname fnsl@\number#1\endcsname{#3}%
}
%    \end{macrocode}
% \end{macro}
%
% \begin{macro}{\@newentry}
% The macros |\new|\meta{type}, where
% \meta{type} $\in \{|acro|, |concept|, |person|\}$ (as described
% below), all call |\@newentry| with an additional first argument,
% namely the numeric \meta{type} identifier of the new entry.
%    \begin{macrocode}
\newcommand\@newentry[4]{%
  \def\@tempa{#2}%
  \edef\@tempb{\@nearverbatim\@tempa}%
  %
  \stepcounter{co@serial}%
  %\PackageWarning{init}{serial counter = \expandafter\theco@serial}%
  \expandafter\xdef\csname fn\number#1@\@tempb\endcsname{%
    \number\theco@serial}%
  \PackageWarning{init}{Def: \meaning\csname fn\number#1@\@tempb\endcsname}%
  \@itxdefineforms{\theco@serial}{#3}{#4}%
}
%    \end{macrocode}
% \end{macro}
%
% \begin{macro}{\newconcept}
%    \begin{macrocode}
\newcommand*\newconcept[3]{%
  \@newentry{0}{#1}{#2}{#3}%
}
%    \end{macrocode}
% \end{macro}
%
% \begin{macro}{\newacronym}
%    \begin{macrocode}
\newcommand*\newacronym[3]{%
  \@newentry{1}{#1}{#2}{#3}%
}
%    \end{macrocode}
% \end{macro}
%
% \begin{macro}{\newperson}
%    \begin{macrocode}
\newcommand*\newperson[3]{%
  \@newentry{2}{#1}{#2}{#3}%
}
%    \end{macrocode}
% \end{macro}
%
% \begin{macro}{\@newentryequiv}
% Note that the macros |\new|\meta{type}|equiv|, where
% \meta{type} $\in \{|acro|, |concept|, |person|\}$ (as described
% below) all call |\@newentryequiv| with an additional first argument,
% namely the numeric \meta{type} identifier of the new entry.
%    \begin{macrocode}
\newcommand*\@newentryequiv[5]{%
  \newif\iffound%
  \def\@tempa{#2}%
  \edef\@tempb{\@nearverbatim\@tempa}%
  \expandafter\ifx\csname fn\number#1 @\@tempb\endcsname\relax%
%    \end{macrocode}
    % Nothing is done if |\csname fn\number#1 @\@tempb\endcsname| is not
    % defined here, but notice that the default value of |\iffound| is
    % \emph{false}.
%    \begin{macrocode}
  \else%
    \foundtrue%
    \edef\co@id{\csname fn\number#1@\@tempb\endcsname}%
    \PackageWarning{init}{Found `\@tempb' (type=\number#1, serial=\co@id)}%
%    \end{macrocode}
    % Store the \emph{short} and the \emph{long}
    % versions of the alias, in that order.
%    \begin{macrocode}
    \stepcounter{co@serial}%
    \@itxdefineforms{\theco@serial}{#3}{#5}%
%    \end{macrocode}
    % Store the numeric id of the concept alias.
%    \begin{macrocode}
    \def\@tempa{#4}%
    \edef\@tempb{\@nearverbatim\@tempa}%
    \expandafter\xdef\csname fn\number#1 e@\@tempb\endcsname{%
      {\co@id}{\theco@serial}}%
  \fi%
  \iffound%
  \else%
    \PackageWarning{InTeX}{Can't find `#2' for sub-concept `#3'}%
  \fi%
}%
%    \end{macrocode}
% \end{macro}
% \begin{macro}{\newconceptequiv}
% \begin{macro}{\newacronymequiv}
% \begin{macro}{\newpersonequiv}
% Explanation of the arguments for the macros |\new|\meta{type}|equiv|, where
% \meta{type} $\in \{|acro|, |concept|, |person|\}$:
%   |#1| identifies the ``parent'' acronym/person,
%   |#2| defines how this equivalent should be typeset in text,
%   |#3| identifies the equivalent form (as in |\co{#3}|), and
%   |#4| represents the spelled-out form of the equivalent.
%    \begin{macrocode}
\newcommand*\newconceptequiv[4]{%
  \@newentryequiv{0}{#1}{#2}{#3}{#4}%
}%
\newcommand*\newacronymequiv[4]{%
  \@newentryequiv{1}{#1}{#2}{#3}{#4}%
}%
\newcommand*\newpersonequiv[4]{%
  \@newentryequiv{2}{#1}{#2}{#3}{#4}%
}%
%    \end{macrocode}
% \end{macro}
% \end{macro}
% \end{macro}
%
% \printindex[raw]
% \Finale
\endinput
