\documentclass[11pt,norsk,english,american,a4paper]{article}
\usepackage{babel}
\usepackage[nomarginplain,nomarginacronym,nomarginperson]{intex}
\usepackage{xspace}
\usepackage[latin1]{inputenc}

\title{Tests of the {\InTeX{}} Indexing Package}

\author{Martin Thorsen Ranang}

\begin{document}
\maketitle

\newacronym{bil}{bil}{boliger i landskap}

\section{Pre-test}

\begin{itemize}
\item \co{A}
\item \co{C}
\end{itemize}

\section{Introduction}
\label{sec:introduction}

Perhaps we should start this journey by talking about \co{bil}.

A good, technically oriented book should be accompanied by a \co{high
quality index}.  Even though research on \co{NLP} constantly leads to
better ways of handling text processing, the problem of creating good
\co{indices} seems to be an \co{AIC} problem.  However, I strongly
believe that a robust framework for creating \co{indices} in
\LaTeX\xspace will help authors avoid easy-to-avoid errors.  Oops.  Do
you know what \co{AIC} is?  It's time we mention \co{mtr} up here
now\ldots

Now that \co{mtr} has been mentioned, we can move on.  And please note
that this has been programmed in \co{\LaTeX}.

\section{Examples}
\label{sec:examples}

It could be interesting to define some \co{IDs}.  \co{MTR} developed
this package to ease the writing of my Ph.D. dissertation.  The area
of interest for \co{MTR}'s research is
 \co{NLP}.  Perhaps I should
invent a new concept, like
 \co{$\alpha$-translation}?  And when I say
\co{$\alpha$-translation}, I mean it.  What happens to a concept like
\co{``fun''}?

The \InTeX\xspace package was written in \LaTeX\xspace and in
\co{Python}.  \co{Python} is a scripting language written by \co{GvR}.

To test the \emph{sort-as} feature, I would like to index \co{Volks Wagen}.



\section{Results}
\label{sec:results}
Well, it seems that \co{MTR} has done something wrong.  \co{mtr}, on
the other hand, seems a bit schizophren.


\section{Next}
\label{sec:next}

And, we keep on writing till we're on the next page.  \co{MTR} is
still with us.


%\index{recursion|see{recursion}}

\printindex[raw]

%\printindex

\end{document}


%%% Local Variables: 
%%% mode: latex
%%% buffer-file-coding-system: latin-1-unix
%%% TeX-master: t
%%% ispell-local-dictionary: "american"
%%% mode: flyspell
%%% End: 
